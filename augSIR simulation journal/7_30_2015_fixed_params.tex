\section*{7/30/2015 - Simulations to assess overshooting behavior with fixed parameters}
\subsection*{Simulation parameters}
\begin{itemize}
	\item Population size: 50, 150, 300
	\item $ \R_0 $: 4, 8
	\item $ \rho: $ 0.1, 0.4
	\item $ \beta =\frac{\R_0}{\text{population size}} $, $ \mu = 1 $
	\item Census interval = 0.05, 0.2
	\item Ten different initializations for each scenario
\end{itemize}
\subsection*{Measures of interest}
\begin{itemize}
	\item Correspondence between true trajectory and augmented population level trajectories
\end{itemize}
\subsection*{Summary of results}
\begin{itemize}
	\item Overshooting behavior is more apparent in larger populations, and when the observation times are further apart from one another.  
\end{itemize}

\subsection*{Next steps and other notes}
\begin{itemize}
	\item The fact that observation times and population size lead to overshooting behavior suggests that the method is sensitive to the amount of missing data. 
	\item Will run Geweke style simulations alternating between simulating X|Y and Y|X, to compare the distribution of X via our method to the distribution of X simulated via Gillespie.  
\end{itemize}
