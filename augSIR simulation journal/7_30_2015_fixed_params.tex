\section*{7/30/2015 - Simulations to assess overshooting behavior with fixed parameters}
\subsection*{Simulation parameters}
\begin{itemize}
	\item Population size: 50, 150, 300
	\item $ \R_0 $: 4, 8
	\item $ \rho: $ 0.1, 0.4
	\item $ \beta =\frac{\R_0}{\text{population size}} $, $ \mu = 1 $
	\item Census interval = 0.05, 0.2
	\item Ten different initializations for each scenario
\end{itemize}
\subsection*{Measures of interest}
\begin{itemize}
	\item Proportion of proposed trajectories accepted
	\item Posterior distributions of model parameters
	\item Complete data log-likelihood
\end{itemize}
\subsection*{Summary of results}
\begin{itemize}
	\item In small populations (50-200ish) the mcmc mixed well and all three initializations settled around roughly the same log-likelihood. The parameters were better recovered in smaller populations.
	\item In larger populations, the chains mixed poorly and were stuck in different modes of the likelihood. 
\end{itemize}

\subsection*{Next steps and other notes}
\begin{itemize}
	\item It was thought that the priors, while perhaps appropriate for each parameter separately, could jointly pull the value of $ \R_0 $ away from the true value. Will explore reparameterizing the model in terms of $ \R_0 $  and sampling parameters using M-H. 
	\item Will run simulations to determine if problems persist with parameters fixed at the true values. 
	\item Will include posterior samples of $ \R_0 $ in future simulations.  
\end{itemize}
